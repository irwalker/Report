\chapter{Conclusions and Future Work}\label{C:us}

This section will discuss the project's contributions, and summarise the scraper and data analysis components. Software development methodology used is then reflected upon. Finally, potential future work identified is outlined.

% \noindent Quantifying user reputation on social media using web-scraping tools was the focus of the project. At its core the project may be split into two elements: the development of a web-scraping framework for social media, and the creation of reputation defining metrics for social media users. 

% At its core the project was successful in two areas: the development of a web-scraping framework for social media, and the creation

% This project looked at quantifying user reputation on social media using web-scraping as the information gathering mechanism. At its core the project was successful in two areas: the development of a web scraping framework for social media, and the creation of reputation defining metrics for social media users. 

% A scraping framework, IHPScrape was developed, and scrapers for Twitter, Facebook and LinkedIn implemented. 

% This section summarises work completed over the project, and reflects on successes with relation to the issues identified at the beginning of the project.

\section{Contributions}
The project's primary contributions were as follows:
% \begin{description}
% 	\item [A web-scraper for Twitter] as part of a wider scraping framework. Existing Twitter scrapers exist, but with performance and usage limitations that are unsuitable for fetching large quantities of data, or indeed reputation-focused data. The Twitter scraper was improved to the order of taking 0.2 seconds on average to fetch a tweet, much better than earlier builds.
% 	
% 	\item [A web-scraping framework] that is suitable for extension onto more social networking sites was created called IHPScrape. The framework abstracts low-level networking details for further works, and reduces development time.
% 	
% 	\item [A set of reputation-inferring policies] for use on Twitter are described, with the potential for more to be created. These exemplar policies are created with reference to observed patterns in data collected.
% 	
% 	\item [A dataset of 1.7 million tweets] and associated tweet meta-data. This dataset was randomly selected. Existing Twitter datasets that were found online were predominantly celebrity based and not useful for research purposes. 
% \end{description}

\begin{description}

	\item [A web-scraper for Twitter] as part of a wider scraping framework was created. Existing Twitter scrapers exist, but with performance and usage limitations that are unsuitable for fetching large quantities of data, or indeed reputation-focused data.
	
	\item [A web-scraping framework] that is suitable for extension onto more social networking sites was made called IHPScrape. The framework abstracts low-level networking details for further works, and reduces development time required for creating scrapers.
	
	\item [A dataset of 1.7 million tweets,] and associated tweet meta-data is created. This dataset was randomly selected. The dataset was found to be rich in semantic potential for analysis, with retweet, temporal and community detection analysis techniques used to generate reputation information. 
	
	\item [A set of reputation-inferring policies] for use on Twitter are described, with the potential for more to be created. These exemplar policies are created with reference to the three analysis techniques discussed.
	
% 	Existing Twitter datasets that were found online were predominantly celebrity based and not useful for research purposes. 
	
	\item[Identification of a Strong Correlation] between Retweet and Favourite counts. This correlation is used as the basis for further policies, by showing that use of retweets and favourites is equivalent for measurement of user impact.
	
	\item [A Twitter Impact Factor] that describes a Twitter user's standing in the community as a function of retweet counts. This builds upon the Hirsch-Index \cite{hirsch2005index} calculation of an academic's standing in the community. The Impact Factor allows for comparison of users at any level of standing, as no upper limit is imposed. Extensions to the \textit{impact factor} were applied through temporal clustering, in order to detect inactivity of users. This extension was qualitatively shown to translate into real-world events for popular users.	
	
	
% 	These exemplar policies are created with reference to observed patterns in collected data.
	
\end{description}

These contributions may be broken into two fundamental sections; the development of a web-scraper, and the creation of reputation defining metrics for social media users. 

%  At its core the project may be split into two elements: the development of a web-scraping framework for social media, and the creation of reputation defining metrics for social media users. 

% This section draws conclusions from the results of the projects and details future work that could be conducted. 

\section{Web-Scraper Implementation}

A social media web-scraping framework was developed. This framework was developed firstly for Twitter and extended onto LinkedIn, validating requirement F3. I was able to demonstrate how web-scraping technologies are suitable for retrieval of basic reputation metrics. Performance of the scraper was greatly increased through incremental builds, with the final build performing a factor of six times better than the previous one.

Unfortunately performance to fetch and parse an entire Twitter profile was shown to be too slow to prove viable in a real-time analytical system. The lower than anticipated performance was due to the dynamic nature of today's social sites, necessitating in potentially hundreds or thousands of separate requests to retrieve an entire profile. Data retrieved was shown to be accurate, withholding some elements which can remain incomplete should their size exceed the threshold selected. Detection avoidance of the framework was also evaluated to have been successful satisfying requirement F4. However once sites detected the scrapers, recovery was not possible due to indistinguishability between detection and the end of a profile. 
% again up until the point of 7,000 + tweets being fetched. 

\section{Data Analysis and Resulting Policies}

A Twitter dataset was created, fetching entire profiles where possible in order to satisfy requirement F1. Data analysis was performed against the Twitter dataset collected of 1.7 million tweets. Retweets were firstly discussed, due to their greater semantic depth than the equivalent \textit{favourite} mechanism. Using retweet counts as a measure of impact, an textit{Impact Factor} calculation is proposed, based upon the Hirsch-Index calculation. This calculation measures productivity and response values on Twitter. Combining this figure with average monthly impact gives a view of average activity, and allows for inactive users to be detected. Different bucketing sizes may be used; in this project, a month was the bucket size of choice. 

Community detection was also experimented with, and the links of retweets and follower/following relationships explored to reveal existence of sub-communities within Twitter. Unfortunately visualisation of entire retweet communities was not achieved due to limitations with the selected visualisation tool. As the MapGenerator tool is currently undergoing improvement, there is potential for improvement in this feature.

Exemplar policies were provided, with respect to the case study of a social forum. Combination of community detection, impact factor, and temporal reputation information were demonstrated to be useful in regards to access policy constructs. These policies address requirement F2, but due to performance constraints of the web scrapers results were not returned in seconds as specified.

\section{Methodology Reflection}

The methodology used combining aspects of Waterfall and Agile methodologies was effective for managing the project. The approach of developing prototypes and iteratively improving these was useful for learning about web scraping technologies and techniques. There were some pitfalls in this approach however. Some effort in developing prototypes went to waste. Facebook was deemed infeasible to scrape due to its interface's constant state of flux. Also during prototyping, Twitter's API changed significantly, rendering much effort lost. 

Adding to the bulk of implementation time was the exploratory nature of the project. New features such as community detection were added late in the project. Fortunately having followed good coding practices such as refactoring, such additions were manageable within the framework developed.  

%  I did experience delays with finishing my implementation and evaluation, as new features such as community detection were added late in the project. Reasons for this late discovery and delay were in part due to the exploratory nature of the project, as will be discussed in the design approach.
 
%  There were some pitfalls in this approach however. Some effort in developing prototypes can go to waste, which occured in this project. Facebook was deemed infeasible to scrape due to its interface's constant state of flux. Also during prototyping, Twitter's API changed significantly, rendering much effort lost.

% Exemplar policies are given, with respect to the example of a social forum. 


% \noindent The evaluation of IHPScrape demonstrates that a web-scraping solution is possible for inferring basic reputation metrics, against the primary case-study of Twitter. However performance to fetch and parse an entire profile is too slow to be viable in a commercial system. The lower than expected performance is due to the dynamic nature of today's social sites, necessitating in potentially hundreds or thousands of separate HTTP requests being sent to fetch an entire profile. 

% However the performance of policies operating on a subset of the full detail from a user's profile is significantly closer to the target aims. Whilst still not achieving response times on the order of a few seconds, when using a subset of tweets greater performance could be achieved with little information loss. Policy effectiveness resulting from data analysis strategies were varying. While the proposed impact factor and temporal clustering methods are possible with the current tools and data, community detection was more limited. The existing visualisation tool cannot handle millions of nodes and even more edges, in this case twitter profiles and their links. However the MapEquation team have acknowledged the deficiencies of their tool, and are working to port this to a non-Flash based system without such limitations. 

% Through qualitative results, retrieved data was shown to be accurate for information such as retweets, favourites and so on. Requirement R6 stated that data should not be missing or incorrect, and this was achieved in the majority of cases. When data was missing, this was due to scrapers being detected and an incomplete list of tweets being returned, for example. Design decisions had to also be made to allow for reasonable performance, such as not fetching the entire list of names who retweeted a post. 

% The framework was shown to be extensible through the implementation of a second social scraper for LinkedIn. The pattern of creating separate classes for pages and defining functions to handle data retrieval related to these pages, and abstracting the actual fetching details was easily extendible to LinkedIn, with the resulting scraper implemented in significantly less time than the original Twitter artefact and framework. I acknowledge however that the comparison is not entirely even, as LinkedIn did not have complications such as Infinite Scrolling. 

% IHPScrape was successful in avoiding detection by sites, and was never blocked entirely running on the University environment. Incomplete profiles dropped below one in ten by the final build. Recovery from detection was not implemented however, as there was no way to distinguish for example the actual end of an individual's twitter page, and the server taking action against a scraper.

% IHPScrape did not fully meet the requirement of remaining flexible despite interface change. The solution had to rely on certain xPath expressions and backend URLs remaining constant, and quantitatively evaluating the exact resistance of the scrapers to change proved infeasible within the timeframe of the project. 

% Experiences with scraping from various social sites lead to conclusions about the suitability of sites for scraping. The more dynamic a site, the more complex and unreliable data fetching will become. 

% Conclusions on which sites are suitable for scraping.


\section{Future Work}

Through this project the following potential extensions were identified as future work.

\subsection{Community Detection Extension}

In its current state, the community detection component of my project only detects community patterns and is not able to associate meaning with these various communities. In order to support a richer set of reputation-defining policies, some form of sentiment or classification of profiles within these communities would be required. In order for more meaningful visualisation of such results, the MapGenerator tool will need to have advanced to a capable level for handling larger amounts of data. As such this was left to future work.

Further, small-world network analysis could be compared to communities detected within Twitter, and my dataset. 

\subsection{Social Media Expansion}

In order to further validate the scraping framework constructed, more social media sites should be explored. This would enable richer still policies and data analysis to be performed. Not all sites would be suitable for scraping, and indeed in some cases the API may be of more use. Sites with more static content are more suitable for web scraping. 

\subsection{Further Evaluation}

Further verification of IHPScrape with respect to resistance to change is required. No Twitter user interface changes were experienced during the course of the project. For the scraper to be applied as part of a legitimate reputation system, verification of its resistance to interface change must be conducted. Comparison against API changes would be more valuable still.

\subsection{Reliability Metric}

Requirement F5 stated that a metric of reliability should be developed based upon quantity of data collected. The performance of policies demonstrated that fetching a full profile for use is not feasible within a real-time system. As such using only a subset of profile data could be more practical, with the attachment of some form of metric pointing to the reliability of data collected.

\subsection{GRAft integration}

As analysis components and policies developed effectively act as a source node in the GRAft model, actual implementation of this could be useful. Community effort would be required to maintain such a solution.