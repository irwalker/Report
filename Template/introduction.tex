\chapter{Introduction}\label{C:int}

Social Networking Sites (SNS) such as \textit{theFacebook} originated in University environments in the early 21st century. Users on social sites are able to create profiles about themselves, and interact with other users. The size of social networking services has grown exponentially with 1.05 billion monthly users on Facebook in December 2012 \cite{fb_users}. Online social platforms such as Facebook, Twitter, Google+ and LinkedIn have served to largely complement real-life interaction. Interactions online may take place with a variety of social spheres; friends, co-workers, family and complete strangers all congregate in one online location \cite{johnson2012facebook}. As these social and business transactions move online, defining who we should and should not trust becomes increasingly important, yet hard to resolve. Although SNS to some extent have inbuilt concepts of trust, using contact networks (Facebook, LinkedIn) and circles (Google+) for example, the concept of trustworthiness is more blurred. Tools to provide a snapshot or hint of an individual's reputation and trustworthiness could be useful. 

% In the digital age many social and business transactions are moving online. As such, defining who we should and should not trust becomes increasingly important, yet hard to resolve. Although SNS to some extent have inbuilt concepts of trust, using contact networks (Facebook, LinkedIn) and circles (Google+) for example, the concept of trustworthiness is more blurred. Tools to provide a snapshot, or hint of an individual's reputation and trustworthiness could be useful. 

Reputation and the task of maintaining one's online reputation is a task often discussed, but with few metrics associated as to what defines an online reputation, or how to evaluate reputation online. Evaluating the standing of users in other communities has been useful for describing access rights \cite{graft_paper}, filtering spam and low quality posts \cite{josang2007survey}, and evaluating the likely quality of feedback on Question and Answer websites \cite{movshovitzanalysis}. This challenge of defining and quantifying reputation was a fundamental motivation for the project.

% This provided a fundamental drive to the project.

% In traditional networking terminology, reputation is a global trust mechanism, as opposed to trust which describes one nodes impression of another in the network. 

% Redo this second paragraph.

% which should influence other members of the networks opinions on the node. 

% Paragraph about social media and how social media has expanded into wide use.

% Paragraph about concepts of reputation. 

% Give a real introduction to the problem domain. What is social media and what is reputation, etc. 

\section{Motivation}

Reputation systems on trading websites have been extremely successful, despite the potential for fraudulent activity \cite{resnick2002trust}. While we might expect a self-aware adult to fairly easily spot equivalent 'fraudulent' activity or deceptive behaviour on social media, many examples have shown this to be untrue. Social crowd sourcing gone wrong has also been a high-profile issue, with incidents such as the Boston Bombings \cite{crowdsourcing_wad,doan2011crowdsourcing,brabham2008crowdsourcing}.

Various strategies have been used to infer the personality or active traits of individuals online; these have been shown to be highly accurate. Applications on SNS have been used to give people an idea of their own personality. The results can then be compared against elements of the person's use of social media. This requires the express permission of the individual concerned to gain any meaningful data, when collecting information through a site's API. While privacy concerns must be taken into account, publicly available information on an individual or company's social media account can be considered fair game. This information could be useful when judging trustworthiness. A new approach, that does not need permission to access information which is publicly-available is needed. 

\section{Proposed Solution}
A solution to the above issues is to use web-scraping strategies to anonymously retrieve reputation data. A more traditional approach would be to retrieve information directly through an API, but few social networks provide sufficient APIs - and they all require express user permission to work. This new system could be used to calculate a person's standing in a community. 

% This system could be used to give a general snapshot about an individual, by using these non-traditional data sources. There are many potential applications and uses for this sort of application.

The aim of this project is to demonstrate the feasibility of a tool that gathers and provides snapshot reputation data from social media sites, using web-scraping as the data gathering method. A framework for scraping social media sites is created called IHPScrape, and is applied to case studies of Twitter, Facebook and LinkedIn. A Twitter dataset is gathered of 1.7 million tweets. Access policies are written based upon the calculated standing of individuals in the community. The Twitter dataset was rich in semantic potential for analysis, and multiple approaches are taken to structure these policies. Retweet and favourite numbers, temporal clustering and community detection are the avenues explored for reputation calculation. 

% Access policies were written based upon the calculated standing of individuals in the community.

% This dataset was found to be rich in semantic potential for analysis, and multiple approaches were taken to structure these policies. 

% By creating such a reputation scraper, we envisage that a general impression about an individual on social media may be constructed. I create a new framework for scraping social media sites called IHPScrape, with the case study of Twitter, Facebook and LinkedIn as examples of the framework's applicability. A Twitter dataset is gathered of 1.7 million tweets, from which reputation metrics and policies are discussed. Finally I evaluate the performance and accuracy of this framework and reputation policies to analyse their potential effectiveness in a larger system.

\section{Contributions}
The primary contributions of this project are:

\begin{description}
	\item [A web-scraper for Twitter] as part of a wider scraping framework. Existing Twitter scrapers exist, but with performance and usage limitations that are unsuitable for fetching large quantities of data, or indeed reputation-focused data.
	
	\item [A web-scraping framework] that is suitable for extension onto more social networking sites is created called IHPScrape. The framework abstracts low-level networking details for further works, and reduces development time.
	
	\item [A dataset of 1.7 million tweets,] and associated tweet meta-data is created. This dataset was randomly selected. Existing Twitter datasets that were found online were predominantly celebrity based and not useful for research purposes. 
	
	\item [A set of reputation-inferring policies] for use on Twitter are described, with the potential for more to be created. These exemplar policies are created with reference to observed patterns in collected data.
	
	\item[Identification of a Strong Correlation] between Retweet and Favourite counts. This correlation is used as the basis for further policies, by showing that use of retweets and favourites is equivalent for measurement of user impact.
	
	\item [A Twitter Impact Factor] that describes a Twitter user's standing in the community as a function of retweet counts. This builds upon the Hirsch-Index \cite{hirsch2005index} calculation of an academic's standing in the community.
	
		
\end{description}

\section{Report Structure}

The remainder of the report is organised as follows. Chapter 2 summarises background work. Software engineering methodologies used throughout the project are described in Chapter 3, followed by the identified requirements in Chapter 4. Chapter 5 describes the design and implementation of the web scraping framework. Data collected using this framework is covered in Chapter 6. Chapter 7 then evaluates the scraping framework created. Finally, Chapter 8 provides conclusions and future work.

% The remainder of the report is structured as follows. Chapter 2 summarises background research that was undertaken in the problem domain. Chapter 3 describes the software engineering methodology used throughout the project. Chapter 4 identifies the specific requirements that the project aims to meet. Chapter 5 describes the design and implementation of the web scraping framework. Chapter 6 analyses data collected and proposes exemplar policies to determine social reputation. Chapter 7 then evaluates the scraping framework created. Chapter 8 provides conclusions and future work.

%The major contribution of the project is to deliver a prototype that gathers online reputation data using web-scraping techniques.The prototype will be integrated with the GRAft reputation system \cite{graft_paper}, in order to link with a person's OpenID\cite{open_id}. Users would be able to generate a reputation value for themselves, or for others they know. 
%explain below sentence!
%It aims to investigate alternative reputation sources, and look at how these can be applied in a practical sense. 

%Provide a scraping framework
%Provide a well-developed and tested scraper for twitter
%Provide a set of reputation policies
%Provide a dataset of 1.7 million tweets, and associated meta-data.