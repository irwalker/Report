\chapter{Introduction}\label{C:us}

The aim of this project is to demonstrate the feasibility of a tool that gathers and provides data from social media websites, using a web-scraping approach. In this report I identify requirements and related work that will help me develop an approach to satisfy them... 

FIXME

\section{The Problem}
In the digital age many social and business transactions are moving online. The size of Facebook and other Social Networking Sites (SNS) has grown exponentially with 1.05 billion monthly users on Facebook in December 2012 \cite{fb_users}. As such, defining who we should and should not treust becomes increasingly important, yet hard to resolve. Although SNS to some extent have inbuilt concepts of trust, using contact networks (Facebook, LinkedIn) and circles (Google+) for example, the concept of trustworthiness is more blurred. Tools to provide a snapshot, or hint of an individual's reputation and trustworthiness could be useful. Reputation is the global trust mechanism, whereas trust is one nodes impression of another in the network.

Reputation systems on trading websites have been extremely succesful, despite the potential for fraudulent activity \cite{}. While we might expect a self-aware adult to fairly easily spot equivilent 'fraudulent' activity or deceptive behaviour on social media, many examples have shown this to be untrue. Social crowdsourcing gone wrong has also been a high-profile issue, with incidents such as the Boston Bombings \cite{}.

Various strategies have been used to infer the personality or active traits of individuals online; these have been shown to be highly accurate. Applications on SNS have been used to give people an idea of their own personality. The results can then be compared against elements of the person's use of social media. This requires the express permission of the individual concerned to gain any meaningful data, when collecting information through a site's API. Which privacy concerns must be taken into account, publicly available information on an individual or company's social media account can be considered fair game. This information could be useful when judging trustworthiness. A new approach, that does not need permission to access information which is publicly-available, is needed. 

\section{Proposed Solution}
A solution to the above issue is to use web-scraping strategies to anonymously retrieve reputation data. A better approach would be to retrieve information directly through an API, but few social networks provide sufficient APIs - and they all require express user permission to work. This system could be used to give a general snapshot about an individual, by using these non-traditional data sources. There are many potential applications and users for this sort of application. Parents might wish to get a glimpse at their child's online friends, or a company might wish to automatically get some information about a potential employee for example.

\section{Contributions}
The major contribution of the project is to deliver a prototype that gathers online reputation data using web-scraping techniques.The prototype will be integrated with the GRAft reputation system \cite{graft_paper}, in order to link with a person's OpenID\cite{open_id}. Users would be able to generate a reputation value for themselves, or for others they know. 
%explain below sentence!
It aims to investigate alternative reputation sources, and look at how these can be applied in a practical sense. 